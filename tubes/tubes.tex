\documentclass[a4paper,titlepage]{article}


\usepackage{a4wide} %smaller borders
\usepackage[utf8]{inputenc}
\usepackage[T1]{fontenc}
\parindent=0pt
\parskip=4pt plus 6pt minus 2pt
\usepackage[siunitx, RPvoltages]{circuitikz}
\usepackage{ctikzmanutils}


\def\Circuitikz{CircuiTi\emph{k}Z}

\author{Jesse op den Brouw\thanks{The Hague University of Applied Sciences (THUAS), \texttt{J.E.J.opdenBrouw@hhs.nl}}}
\title{\Circuitikz\ -- Electronic Tubes}
\date{\today}


%%
%% tubes-circuitikz.tex - test file for tubes in CircuiTiKZ
%%
%% Copyright (c)2019, Jesse E. J. op den Brouw
%%

%% This work may be distributed and/or modified under the
%% conditions of the LaTeX Project Public License, either version 1.3c
%% of this license or (at your option) any later version.
%% The latest version of this license is in
%%   http://www.latex-project.org/lppl.txt
%% and version 1.3c or later is part of all distributions of LaTeX 
%% version 2003/12/01 or later.

%% This work consists of the files tubes-circuitikz.tex,
%% tubes.tex, tubes-speedtest.tex

%% This software is provided 'as is', without warranty of any kind,
%% either expressed or implied, including, but not limited to, the
%% implied warranties of merchantability and fitness for a
%% particular purpose.

%% Jesse op den Brouw
%% Department of Electrical Engineering
%% The Hague University of Applied Sciences
%% Rotterdamseweg 137, 2628 AL, Delft
%% Netherlands
%% J.E.J.opdenBrouw@hhs.nl

%% The newest version of this document should always be available
%% from https://github.com/jesseopdenbrouw/circuitikz_enhancement

%% Version 1.0
%%

%%%%%%%%%%%%%%%%%%%%%%%%%%%%%%%%%%%%%%%%%%%%%%%%%%%%%%%%%%%%%%%%%%%%%%%%%%%%%%%
%%
%% CIRCUITIKZ - TUBES: DIODE, TRIODE, TETRODE and PENTODE
%%
%% (c)2019, J.E.J. op den Brouw <J.E.J.opdenBrouw@hhs.nl>
%%
%%%%%%%%%%%%%%%%%%%%%%%%%%%%%%%%%%%%%%%%%%%%%%%%%%%%%%%%%%%%%%%%%%%%%%%%%%%%%%%

\makeatletter


\pgfdeclareshape{mycircle}
{
    \savedanchor\centerpoint{
        \pgf@x = .5\wd\pgfnodeparttextbox
        \pgf@y = .5\ht\pgfnodeparttextbox
    }
    \anchor{center}{\centerpoint}

    \backgroundpath{
        \pgfnode{triode}{center}{}{}{\pgfusepath{draw}}
    }

}

\makeatother


\begin{document}




\begin{groupdesc}
    \circuitdesc*{magnetron}{Magnetron}{}( anode/-90/0.2, cathode1/135/0.2,
    cathode2/45/0.2, left/180/0.2, right/0/0.2, top/90/0.4 )
\end{groupdesc}

\begin{groupdesc}
	\circuitdesc*{diodetube}{Tube Diode}{} ( anode/90/0.2, cathode/-90/0.2 )
	\circuitdesc*{triode}{Triode}{} ( anode/90/0.2, cathode/-90/0.2, grid/180/0.2 )
	\circuitdesc*{tetrode}{Tetrode}{} ( anode/90/0.2, cathode/-90/0.2, grid/180/0.2,screen/180/0.2 )
	\circuitdesc*{pentode}{Pentode}{} ( anode/90/0.2, cathode/-90/0.2, grid/180/0.2,screen/180/0.2,suppressor/180/0.2 )
\end{groupdesc}

Normally, the filament is not drawn and there are no anchors present. If you want a filament, put the \verb|filament| option in the node description:

\begin{groupdesc}
	\circuitdesc*{diodetube,filament}{Tube Diode}{} ( anode/90/0.2, cathode/-90/0.2 )
\end{groupdesc}

Sometimes, you don't want the cathode to be drawn (but you do want the filament). Use the \verb|nocathode| option in the node description:

\begin{groupdesc}
	\circuitdesc*{diodetube,filament,nocathode}{Tube Diode}{} ( anode/90/0.2, cathode/-90/0.2 )
\end{groupdesc}

Example triode amplifier:

\begin{lstlisting}
\draw (0,0) node (start) {}
            	to[sV=$V_i$] ++(0,2+\ctikzvalof{tubes/triode/height})
            	to[C=$C_i$] ++(2,0) node (Rg) {}
				to[R=$R_g$] (Rg |- start)
(Rg)			to[short,*-] ++(1,0)
				node[triode,anchor=grid] (Tri) {} ++(2,0)
(Tri.cathode)	to[R=$R_c$,-*] (Tri.cathode |- start)
(Tri.anode)		to [R=$R_a$] ++(0,2)
				to [short] ++(3.5,0) node(Vatop) {}
				to [V<=$V_a$] (Vatop |- start)
				to [short] (start)
(Tri.anode) ++(0,0.2) to[C=$C_o$,*-o] ++(2,0)
(Tri.cathode) ++(0,-0.2) to[short,*-] ++(1.5,0) node(Cctop) {}
				to[C=$C_c$,-*] (start -| Cctop)
;
\draw[red,thin] (Tri.north west) rectangle (Tri.south east);
\draw[blue] (Tri.east) node[right] {$\frac{1}{2}$ECC83};
\end{lstlisting}
\begin{tikzpicture}
\draw (0,0) node (start) {}
            	to[sV=$V_i$] ++(0,2+\ctikzvalof{tubes/triode/height})
            	to[C=$C_i$] ++(2,0) node (Rg) {}
				to[R=$R_g$] (Rg |- start)
(Rg)			to[short,*-] ++(1,0)
				node[triode,anchor=grid] (Tri) {} ++(2,0)
(Tri.cathode)	to[R=$R_c$,-*] (Tri.cathode |- start)
(Tri.anode)		to [R=$R_a$] ++(0,2)
				to [short] ++(3.5,0) node(Vatop) {}
				to [V<=$V_a$] (Vatop |- start)
				to [short] (start)
(Tri.anode) ++(0,0.2) to[C=$C_o$,*-o] ++(2,0)
(Tri.cathode) ++(0,-0.2) to[short,*-] ++(1.5,0) node(Cctop) {}
				to[C=$C_c$,-*] (start -| Cctop)
;
\draw[red,thin] (Tri.north west) rectangle (Tri.south east);
\draw[blue] (Tri.east) node[right] {$\frac{1}{2}$ECC83};
\end{tikzpicture}

\end{document}