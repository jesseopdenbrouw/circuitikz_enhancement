\documentclass[a4paper,titlepage]{article}


\usepackage{a4wide} %smaller borders
\usepackage[utf8]{inputenc}
\usepackage[T1]{fontenc}
\parindent=0pt
\parskip=4pt plus 6pt minus 2pt
\usepackage[siunitx, RPvoltages]{circuitikz}
\usepackage{ctikzmanutils}


\def\Circuitikz{CircuiTi\emph{k}Z}

\author{Jesse op den Brouw\thanks{The Hague University of Applied Sciences (THUAS), \texttt{J.E.J.opdenBrouw@hhs.nl}}}
\title{\Circuitikz\ -- Triode}
\date{\today}




%%%%%%%%%%%%%%%%%%%%%%%%%%%%%%%%%%%%%%%%%%%%%%%%%%%%%%%%%%%%%%%%%%%%%%%%%%%%%%%
%%
%% CIRCUITIKZ - TRIODE WITH FILAMENT
%%
%%%%%%%%%%%%%%%%%%%%%%%%%%%%%%%%%%%%%%%%%%%%%%%%%%%%%%%%%%%%%%%%%%%%%%%%%%%%%%%
%%
%% Triode keys
%%
\ctikzset{tubes/width/.initial=1}                           % relative width
\ctikzset{tubes/height/.initial=1}                          % relative height

\ctikzset{tubes/triode/thickness/.initial=1}                % relative line thickness
\ctikzset{tubes/triode/tube width/.initial=0.40}            % radius of tube circle
\ctikzset{tubes/triode/anode distance/.initial=0.20}        % distance from grid
\ctikzset{tubes/triode/anode width/.initial=0.20}           % width of (half) an anode/plate
\ctikzset{tubes/triode/cathode distance/.initial=0.20}      % distance from grid
\ctikzset{tubes/triode/cathode width/.initial=0.20}         % width of (half) an cathode
\ctikzset{tubes/triode/cathode corners/.initial=0.06}       % corners of the cathode wire
\ctikzset{tubes/triode/cathode right extend/.initial=0.075} % extension at the right side
\ctikzset{tubes/triode/grid protrusion/.initial=0.10}       % distance in tube circle
\ctikzset{tubes/triode/grid dashes/.initial=5}              % number of grid dashes
\ctikzset{tubes/triode/filament distance/.initial=0.05}     % distance from cathode
\ctikzset{tubes/triode/filament angle/.initial=15}          % Angle from centerpoint


%%
%% Triode filament behaves like a sort of style
%%
\makeatletter
\newif\ifpgf@circuit@triode@filament
\pgfkeys{/tikz/filament/.add code={}{\pgf@circuit@triode@filamenttrue}}
\ctikzset{tubes/triode/filament/.add code={}{\pgf@circuit@triode@filamenttrue}}
\makeatother

%%
%% This is the Circuitikz element for a triode filament.
%% Basic anchors: grid, anode, cathode, text
%%
\makeatletter
\pgfdeclareshape{triode}{
    \anchor{center}{
        \pgfpointorigin
    }
    \savedanchor\northwest{%
		\ifdim\ctikzvalof{tubes/width}\pgf@circ@Rlen>\ctikzvalof{tubes/height}\pgf@circ@Rlen\relax
			\pgf@circ@res@up=\ctikzvalof{tubes/width}\pgf@circ@Rlen
		\else
			\pgf@circ@res@up=\ctikzvalof{tubes/height}\pgf@circ@Rlen
		\fi
		% x and y should be half the Rlen
        \pgf@y=\pgf@circ@res@up
        \pgf@y=.5\pgf@y
        \pgf@x=-\pgf@circ@res@up
        \pgf@x=-\pgf@circ@res@up
        \pgf@x=.5\pgf@x
    }
	\anchor{north} {%
		\northwest
		\pgf@x=0pt
	}
    \anchor{east}{%
        \northwest
        \pgf@x=-\pgf@x
        \pgf@y=0pt
    }
	\anchor{south}{%
        \northwest
        \pgf@y=-\pgf@y
		\pgf@x=0pt
	}
    \anchor{west}{%
        \northwest
        \pgf@y=0pt
    }
    \anchor{north west}{%
        \northwest
    }
    \anchor{north east}{%
        \northwest
        \pgf@x=-\pgf@x
    }
    \anchor{south east}{
        \northwest
        \pgf@x=-\pgf@x
        \pgf@y=-\pgf@y
    }
    \anchor{south west}{
        \northwest
        \pgf@y=-\pgf@y
    }
	\anchor{anode} {%
		\northwest
		\pgf@x=0pt
	}
    \anchor{grid}{%
        \northwest
        \pgf@y=0pt
    }
	\anchor{cathode}{%
        \northwest
        \pgf@y=-\pgf@y
		\pgf@x=2\pgf@x
		\pgf@x=\ctikzvalof{tubes/triode/cathode width}\pgf@x
	}
    \anchor{text}{%
        \northwest
        \pgf@x=-\pgf@x
        \pgf@y=0pt
    }
	\backgroundpath{
		\pgfscope
			\ifdim\ctikzvalof{tubes/width}\pgf@circ@Rlen>\ctikzvalof{tubes/height}\pgf@circ@Rlen\relax
				\pgf@circ@res@up=\ctikzvalof{tubes/width}\pgf@circ@Rlen
			\else
				\pgf@circ@res@up=\ctikzvalof{tubes/height}\pgf@circ@Rlen
			\fi
			\pgfsetlinewidth{\ctikzvalof{tubes/triode/thickness}\pgflinewidth}

			% Tube fill color (if any)
			\ifx\tikz@fillcolor\pgfutil@empty
			\else
				\pgfscope
					\pgfsetfillcolor{\tikz@fillcolor}
					\pgfpathmoveto{\pgfpointorigin}
					\pgfpathcircle{\pgfpointorigin}{\ctikzvalof{tubes/triode/tube width}\pgf@circ@res@up}
					\pgfusepath{fill}
				\endpgfscope
    		\fi

			% Tube (circle)
			\pgfpathmoveto{\pgfpointorigin}
			\pgfpathcircle{\pgfpointorigin}{\ctikzvalof{tubes/triode/tube width}\pgf@circ@res@up}

			% Anode (or plate)
			\pgfpathmoveto{\pgfpoint{0pt}{0.5\pgf@circ@res@up}} % north
			\pgfpathlineto{\pgfpoint{0pt}{\ctikzvalof{tubes/triode/anode distance}\pgf@circ@res@up}}
			\pgfpathmoveto{\pgfpoint{-\ctikzvalof{tubes/triode/anode width}\pgf@circ@res@up}{\ctikzvalof{tubes/triode/anode distance}\pgf@circ@res@up}}
			\pgfpathlineto{\pgfpoint{\ctikzvalof{tubes/triode/anode width}\pgf@circ@res@up}{\ctikzvalof{tubes/triode/anode distance}\pgf@circ@res@up}}

			% Grid protrusion
			\pgfpathmoveto{\pgfpoint{-.50\pgf@circ@res@up}{0pt}}
			\pgfmathqparse{-\ctikzvalof{tubes/triode/tube width}\pgf@circ@res@up+\ctikzvalof{tubes/triode/grid protrusion}\pgf@circ@res@up}
			\pgfpathlineto{\pgfpoint{\pgfmathresult pt}{0pt}}

			% Grid dashes: calculations
			\pgf@xa=\pgfmathresult pt
			\pgf@xb=\ctikzvalof{tubes/triode/tube width}\pgf@circ@res@up
			\@tempcnta=\ctikzvalof{tubes/triode/grid dashes}  % dashes*2+1
			\multiply\@tempcnta by 2\relax
			\advance\@tempcnta by 1\relax
			\pgf@circ@res@step=\pgf@xb % calculate step
			\advance\pgf@circ@res@step by -\pgf@xa
			\divide\pgf@circ@res@step by \@tempcnta
			% Grid dashes: draw
			\pgf@circ@res@temp=\pgf@xa
			\@tempcnta=\ctikzvalof{tubes/triode/grid dashes}  % dashes
			\loop
				\advance\pgf@circ@res@temp by\pgf@circ@res@step
				\pgfpathmoveto{\pgfpoint{\pgf@circ@res@temp}{0pt}}
				\advance\pgf@circ@res@temp by\pgf@circ@res@step
				\pgfpathlineto{\pgfpoint{\pgf@circ@res@temp}{0pt}}
				\advance\@tempcnta by-1
				\ifnum\@tempcnta>0\relax
			\repeat

			% Filament (is not drawn by default)
			\ifpgf@circuit@triode@filament
				\pgf@circ@res@temp=-\ctikzvalof{tubes/triode/cathode distance}\pgf@circ@res@up
				\advance\pgf@circ@res@temp by -\ctikzvalof{tubes/triode/filament distance}\pgf@circ@res@up
				\pgfmathparse{(\ctikzvalof{tubes/triode/tube width}*sin(\ctikzvalof{tubes/triode/filament angle})}
				\pgfextra{\pgfmathresult}
				\pgf@xa=\pgfmathresult\pgf@circ@res@up
				\pgfmathparse{\ctikzvalof{tubes/triode/tube width}*cos(\ctikzvalof{tubes/triode/filament angle})}
				\pgfextra{\pgfmathresult}
				\pgf@ya=\pgfmathresult\pgf@circ@res@up
				\pgfpathmoveto{\pgfpoint{0pt}{\pgf@circ@res@temp}}
				\pgfpathlineto{\pgfpoint{-\pgf@xa}{-\pgf@ya}}
				\pgfpathlineto{\pgfpoint{-\pgf@xa}{-0.5\pgf@circ@res@up}}
				\pgf@circuit@triode@filamentfalse
				\pgfpathmoveto{\pgfpoint{0pt}{\pgf@circ@res@temp}}
				\pgfpathlineto{\pgfpoint{\pgf@xa}{-\pgf@ya}}
				\pgfpathlineto{\pgfpoint{\pgf@xa}{-0.5\pgf@circ@res@up}}
				\pgf@circuit@triode@filamentfalse
			\fi

			% Cathode
			\pgfsetcornersarced{\pgfpoint{\ctikzvalof{tubes/triode/cathode corners}\pgf@circ@res@up}{\ctikzvalof{tubes/triode/cathode corners}\pgf@circ@res@up}}
			\pgfpathmoveto{\pgfpoint{-\ctikzvalof{tubes/triode/cathode width}\pgf@circ@res@up}{-.5\pgf@circ@res@up}} % cathode anchor
			\pgfpathlineto{\pgfpoint{-\ctikzvalof{tubes/triode/cathode width}\pgf@circ@res@up}{-\ctikzvalof{tubes/triode/cathode distance}\pgf@circ@res@up}}
			\pgfpathlineto{\pgfpoint{\ctikzvalof{tubes/triode/cathode width}\pgf@circ@res@up}{-\ctikzvalof{tubes/triode/cathode distance}\pgf@circ@res@up}}
			\pgfpathlineto{\pgfpoint{\ctikzvalof{tubes/triode/cathode width}\pgf@circ@res@up}{-\ctikzvalof{tubes/triode/cathode distance}\pgf@circ@res@up-\ctikzvalof{tubes/triode/cathode right extend}\pgf@circ@res@up}}
			\pgfusepath{draw}
			            
		\endpgfscope
	}
}
\makeatother


\begin{document}


This document describes the use of a standard triode tube in the schematic drawing extension \Circuitikz. It is highly configurable and has a large number of anchors.

\subsubsection*{Basic usage}

The triode must be placed as a node, for example:

\begin{LTXexample}[varwidth]
\begin{tikzpicture}
\draw (0,0) node[triode] (Tri) {};
\end{tikzpicture}
\end{LTXexample}

You can draw the filament if you like:

\begin{LTXexample}[varwidth]
\begin{tikzpicture}line width=1pt,font=\sffamily\footnotesize
\draw (0,0) node[triode,filament] (Tri) {};
\end{tikzpicture}
\end{LTXexample}

\subsubsection*{Triode anchors}

The triode has three basic anchors and a text (or label) anchor: anode, cathode grid and text. The can be accessed by \emph{nodename}.\texttt{anode}, \emph{nodename}.\texttt{cathode}, \emph{nodename}.\texttt{grid} and \emph{nodename}.\texttt{text}. The filament has no anchors. The position of the anchors is shown below.

\begin{LTXexample}[varwidth]
\begin{tikzpicture}
\draw (0,0) node[triode] (Tri) {}
\foreach \tria/\trip in {grid/left, anode/above,
                         cathode/below, text/right}
{
   (Tri.\tria) node[circ] {} node[\trip] {\tria}
}
;
\end{tikzpicture}
\end{LTXexample}

Below is a figure that shows all geographical anchors (scaled by 2, filled with pink, bounding box).

\ctikzset{tubes/triode/tube width/.initial=0.50}            % radius of tube circle

\begin{tikzpicture}
\draw (0,0) node[triode,scale=2,filament,fill=pink!50] (Tri) {}; % center
\draw[red] (Tri.north west) rectangle (Tri.south east);

\draw
\foreach \tria/\trip in {center/above,west/left,north/above,south/below,east/right,north west/left, north east/right, south west/left, south east/right} {
	(Tri.\tria) node[circ] {} node[\trip] {\tria}
}
;
\end{tikzpicture}

The following parameters can be altered:

\begin{lstlisting}[basicstyle=\small\ttfamily]
\ctikzset{tubes/triode/tube width/.initial=0.40}            % radius of tube circle
\ctikzset{tubes/triode/anode distance/.initial=0.20}        % distance from grid
\ctikzset{tubes/triode/anode width/.initial=0.20}           % width of (half) an anode/plate
\ctikzset{tubes/triode/cathode distance/.initial=0.20}      % distance from grid
\ctikzset{tubes/triode/cathode width/.initial=0.20}         % width of (half) an cathode
\ctikzset{tubes/triode/cathode corners/.initial=0.06}       % corners of the cathode wire
\ctikzset{tubes/triode/cathode right extend/.initial=0.075} % extension at the right side
\ctikzset{tubes/triode/grid protrusion/.initial=0.10}       % distance in tube circle
\ctikzset{tubes/triode/grid dashes/.initial=5}              % number of grid dashes
\ctikzset{tubes/triode/filament distance/.initial=0.05}     % distance from cathode
\ctikzset{tubes/triode/filament angle/.initial=15}          % Angle from centerpoint
\end{lstlisting}


\bigskip

Sample network:

\ctikzset{tripoles/thickness=1}
\ctikzset{bipoles/thickness=1}

\begin{tikzpicture}[line width=1]
\draw (0,0) node (start) {}
            	to[sV=$V_i$] ++(0,2+\ctikzvalof{tubes/height})
            	to[C=$C_i$] ++(2,0)
				node[triode,anchor=grid,filament] (Tri) {} ++(2,0)
(Tri.cathode)	to[R=$R_c$,-*] (Tri.cathode |- start)
(Tri.anode)		to [R=$R_a$] ++(0,2)
				to [short] ++(3.5,0) node(Vatop) {}
				to [V<=$V_a$] (Vatop |- start)
				to [short] (start)
(Tri.anode) ++(0,0.2) to[C=$C_o$,*-o] ++(2,0)
(Tri.cathode) ++(0,-0.2) to[short,*-] ++(1.5,0) node(Cctop) {}
				to[C=$C_c$,-*] (start -| Cctop)
;

\draw[red,thin] (Tri.north west) rectangle (Tri.south east);
\draw[blue] (Tri.text) node[right] {$\frac{1}{2}$ECC83};
\end{tikzpicture}

\end{document}