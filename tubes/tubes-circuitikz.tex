%%
%% tubes-circuitikz.tex - test file for tubes in CircuiTiKZ
%%
%% Copyright (c)2019, Jesse E. J. op den Brouw
%%

%% This work may be distributed and/or modified under the
%% conditions of the LaTeX Project Public License, either version 1.3c
%% of this license or (at your option) any later version.
%% The latest version of this license is in
%%   http://www.latex-project.org/lppl.txt
%% and version 1.3c or later is part of all distributions of LaTeX 
%% version 2003/12/01 or later.

%% This work consists of the files tubes-circuitikz.tex,
%% tubes.tex, tubes-speedtest.tex

%% This software is provided 'as is', without warranty of any kind,
%% either expressed or implied, including, but not limited to, the
%% implied warranties of merchantability and fitness for a
%% particular purpose.

%% Jesse op den Brouw
%% Department of Electrical Engineering
%% The Hague University of Applied Sciences
%% Rotterdamseweg 137, 2628 AL, Delft
%% Netherlands
%% J.E.J.opdenBrouw@hhs.nl

%% The newest version of this document should always be available
%% from https://github.com/jesseopdenbrouw/circuitikz_enhancement

%% Version 1.0
%%

%%%%%%%%%%%%%%%%%%%%%%%%%%%%%%%%%%%%%%%%%%%%%%%%%%%%%%%%%%%%%%%%%%%%%%%%%%%%%%%
%%
%% CIRCUITIKZ - TUBES: DIODE, TRIODE, TETRODE and PENTODE
%%
%% (c)2019, J.E.J. op den Brouw <J.E.J.opdenBrouw@hhs.nl>
%%
%%%%%%%%%%%%%%%%%%%%%%%%%%%%%%%%%%%%%%%%%%%%%%%%%%%%%%%%%%%%%%%%%%%%%%%%%%%%%%%

%%
%% Tube keys
%%
\ctikzset{tubes/width/.initial=1}                    % relative width
\ctikzset{tubes/height/.initial=1.4}                 % relative height
\ctikzset{tubes/tube radius/.initial=0.40}           % radius of tube circle
\ctikzset{tubes/anode distance/.initial=0.40}        % distance from center
\ctikzset{tubes/anode width/.initial=0.40}           % width of an anode/plate
\ctikzset{tubes/grid protrusion/.initial=0.25}       % distance from center
\ctikzset{tubes/grid dashes/.initial=5}              % number of grid dashes
\ctikzset{tubes/grid separation/.initial=0.2}        % separation between grids
\ctikzset{tubes/grid shift/.initial=0.0}             % y shift grids from center
\ctikzset{tubes/cathode distance/.initial=0.40}      % distance from grid
\ctikzset{tubes/cathode width/.initial=0.40}         % width of an cathode
\ctikzset{tubes/cathode corners/.initial=0.06}       % corners of the cathode wire
\ctikzset{tubes/cathode right extend/.initial=0.075} % extension at the right side
\ctikzset{tubes/filament distance/.initial=0.1}      % distance from cathode
\ctikzset{tubes/filament angle/.initial=15}          % Angle from centerpoint


%%
%% Tube filament, nocathode and fullcathode node options
%%
\makeatletter
\newif\ifpgf@circuit@tubes@filament\pgf@circuit@tubes@filamentfalse
\pgfkeys{/tikz/filament/.add code={}{\pgf@circuit@tubes@filamenttrue}}
\ctikzset{tubes/filament/.add code={}{\pgf@circuit@tubes@filamenttrue}}
\newif\ifpgf@circuit@tubes@nocathode\pgf@circuit@tubes@nocathodefalse
\pgfkeys{/tikz/nocathode/.add code={}{\pgf@circuit@tubes@nocathodetrue}}
\ctikzset{tubes/nocathode/.add code={}{\pgf@circuit@tubes@nocathodetrue}}
\newif\ifpgf@circuit@tubes@fullcathode\pgf@circuit@tubes@fullcathodefalse
\pgfkeys{/tikz/fullcathode/.add code={}{\pgf@circuit@tubes@fullcathodetrue}}
\ctikzset{tubes/fullcathode/.add code={}{\pgf@circuit@tubes@fullcathodetrue}}

\pgfkeys{/tikz/model/.initial={}}

% Draw tube outline
\def\pgf@circ@tubes@drawtube{%
	\pgfpathmoveto{\pgfpoint{-\pgf@circ@res@right}{0pt}}
	\pgfutil@tempdima=\pgf@circ@res@up
	\advance\pgfutil@tempdima by -\pgf@circ@res@right
	\pgfpathlineto{\pgfpoint{-\pgf@circ@res@right}{\pgfutil@tempdima}}
	\pgfpatharc{180}{0}{\pgf@circ@res@right}
	\pgfpathlineto{\pgfpoint{\pgf@circ@res@right}{-\pgfutil@tempdima}}
	\pgfpatharc{180}{0}{-\pgf@circ@res@right}
	\pgfpathclose
}


%% The diode (tube), triode, tetrode and pentode only differ in the
%% number of grids. So we construct a generic declare function in
%% which we can put code for the grid anchors and grid drawing code
%% \pgfcircdeclaretube{tube name}{grid anchors}{grid drawing code}
\long\def\pgfcircdeclaretube#1#2#3{%
	%\nodeparts{}
	\pgfdeclareshape{#1}{
	    \anchor{center}{
	        \pgfpointorigin
	    }
	    \savedanchor\northwest{%
	        \pgf@circ@res@up=\ctikzvalof{tubes/height}\pgf@circ@Rlen
			\pgf@circ@res@right=\ctikzvalof{tubes/width}\pgf@circ@Rlen
			% x and y should be half the Rlen
	        \pgf@y=\pgf@circ@res@up
	        \pgf@y=.5\pgf@y
	        \pgf@x=-\pgf@circ@res@right
	        \pgf@x=.5\pgf@x
	    }
		\anchor{north} {%
			\northwest
			\pgf@x=0pt
		}
	    \anchor{east}{%
	        \northwest
	        \pgf@x=-\pgf@x
	        \pgf@y=0pt
	    }
		\anchor{south}{%
	        \northwest
	        \pgf@y=-\pgf@y
			\pgf@x=0pt
		}
	    \anchor{west}{%
	        \northwest
	        \pgf@y=0pt
	    }
	    \anchor{north west}{%
	        \northwest
	    }
	    \anchor{north east}{%
	        \northwest
	        \pgf@x=-\pgf@x
	    }
	    \anchor{south east}{
	        \northwest
	        \pgf@x=-\pgf@x
	        \pgf@y=-\pgf@y
	    }
	    \anchor{south west}{
	        \northwest
	        \pgf@y=-\pgf@y
	    }
		\anchor{anode} {%
			\northwest
			\pgf@x=0pt
		}
		\anchor{cathode}{%
			\northwest
			\pgf@y=-\pgf@y
			\pgf@x=\ctikzvalof{tubes/cathode width}\pgf@x
		}
		\anchor{cathode 1}{%
			\northwest
			\pgf@y=-\pgf@y
			\pgf@x=\ctikzvalof{tubes/cathode width}\pgf@x
		}
		\anchor{cathode 2}{%
			\northwest
			\pgf@y=-\pgf@y
			\pgf@x=-\ctikzvalof{tubes/cathode width}\pgf@x
		}
		\anchor{filament 1}{%
			\northwest
			\pgfmathparse{(\ctikzvalof{tubes/tube radius}*sin(\ctikzvalof{tubes/filament angle})}
			\pgf@x=\pgfmathresult\pgf@x
			\pgf@y=-\pgf@y
		}
		\anchor{filament 2}{%
			\northwest
			\pgfmathparse{(\ctikzvalof{tubes/tube radius}*sin(\ctikzvalof{tubes/filament angle})}
			\pgf@x=-\pgfmathresult\pgf@x
			\pgf@y=-\pgf@y
		}
%	    \savedanchor{model}{% east anchor
%	        \northwest
%	        \pgf@x=-\pgf@x
%	        \pgf@y=0pt
%	    }

		% Extra anchors
		#2

		\backgroundpath{
			\pgfscope
				% Line width for tripoles
				\pgfsetlinewidth{\ctikzvalof{tripoles/thickness}\pgflinewidth}
	
				% Setup to draw tube
				\pgf@circ@res@up=\ctikzvalof{tubes/height}\pgf@circ@Rlen
				\pgf@circ@res@right=\ctikzvalof{tubes/width}\pgf@circ@Rlen
				\pgf@circ@res@up=\ctikzvalof{tubes/tube radius}\pgf@circ@res@up
				\pgf@circ@res@right=\ctikzvalof{tubes/tube radius}\pgf@circ@res@right
	
				% Tube fill color (if any)
				\ifx\tikz@fillcolor\pgfutil@empty
				\else
					\pgfscope
						\pgfsetfillcolor{\tikz@fillcolor}
						\pgf@circ@tubes@drawtube
						\pgfusepath{fill}
					\endpgfscope
	    		\fi
	
				% Tube outline
				\pgf@circ@tubes@drawtube

				% Setup to draw grid, filament, anode and cathode
				\pgf@circ@res@up=\ctikzvalof{tubes/height}\pgf@circ@Rlen
				\pgf@circ@res@right=\ctikzvalof{tubes/width}\pgf@circ@Rlen
				\pgf@circ@res@up=0.5\pgf@circ@res@up
				\pgf@circ@res@right=0.5\pgf@circ@res@right

				% Grid drawing
				#3
				
				% Filament (is not drawn by default)
				\ifpgf@circuit@tubes@filament
					\pgf@circ@res@temp=-\ctikzvalof{tubes/cathode distance}\pgf@circ@res@up
					\advance\pgf@circ@res@temp by -\ctikzvalof{tubes/filament distance}\pgf@circ@res@up
					\pgfmathparse{(\ctikzvalof{tubes/tube radius}*sin(\ctikzvalof{tubes/filament angle})}
					\pgf@xa=\pgfmathresult\pgf@circ@res@right
					\pgfmathparse{\ctikzvalof{tubes/tube radius}+\ctikzvalof{tubes/tube radius}*cos(\ctikzvalof{tubes/filament angle}}
					\pgf@ya=\pgfmathresult\pgf@circ@res@up
					\pgfpathmoveto{\pgfpoint{0pt}{\pgf@circ@res@temp}}
					\pgfpathlineto{\pgfpoint{-\pgf@xa}{-\pgf@ya}}
					\pgfpathlineto{\pgfpoint{-\pgf@xa}{-\pgf@circ@res@up}}
					\pgfpathmoveto{\pgfpoint{0pt}{\pgf@circ@res@temp}}
					\pgfpathlineto{\pgfpoint{\pgf@xa}{-\pgf@ya}}
					\pgfpathlineto{\pgfpoint{\pgf@xa}{-\pgf@circ@res@up}}
					\pgf@circuit@tubes@filamentfalse
				\fi
	
				% Anode (or plate)
				\pgfpathmoveto{\pgfpoint{0pt}{\pgf@circ@res@up}} % north
				\pgfpathlineto{\pgfpoint{0pt}{\ctikzvalof{tubes/anode distance}\pgf@circ@res@up}}
				\pgfpathmoveto{\pgfpoint{-\ctikzvalof{tubes/anode width}\pgf@circ@res@right}{\ctikzvalof{tubes/anode distance}\pgf@circ@res@up}}
				\pgfpathlineto{\pgfpoint{\ctikzvalof{tubes/anode width}\pgf@circ@res@right}{\ctikzvalof{tubes/anode distance}\pgf@circ@res@up}}
	
				% Cathode
				\ifpgf@circuit@tubes@nocathode
					\pgf@circuit@tubes@nocathodefalse
				\else
					\pgfsetcornersarced{\pgfpoint{\ctikzvalof{tubes/cathode corners}\pgf@circ@res@up}{\ctikzvalof{tubes/cathode corners}\pgf@circ@res@up}}
					\pgfpathmoveto{\pgfpoint{-\ctikzvalof{tubes/cathode width}\pgf@circ@res@right}{-\pgf@circ@res@up}}
					\pgfpathlineto{\pgfpoint{-\ctikzvalof{tubes/cathode width}\pgf@circ@res@right}{-\ctikzvalof{tubes/cathode distance}\pgf@circ@res@up}}
					\pgfpathlineto{\pgfpoint{\ctikzvalof{tubes/cathode width}\pgf@circ@res@right}{-\ctikzvalof{tubes/cathode distance}\pgf@circ@res@up}}
					\ifpgf@circuit@tubes@fullcathode
						\pgfpathlineto{\pgfpoint{\ctikzvalof{tubes/cathode width}\pgf@circ@res@right}{-\pgf@circ@res@up}}
						\pgf@circuit@tubes@fullcathodefalse
					\else
						\pgfpathlineto{\pgfpoint{\ctikzvalof{tubes/cathode width}\pgf@circ@res@right}{-\ctikzvalof{tubes/cathode distance}\pgf@circ@res@up-\ctikzvalof{tubes/cathode right extend}\pgf@circ@res@up}}
					\fi
				\fi

				% Draw the background	
				\pgfusepath{draw}

				% Put the model on the east side, if user supplied it
				\def\pgf@temp{}
				\edef\pgfutil@temp{\pgfkeysvalueof{/tikz/model}}
				\ifx\pgf@temp\pgfutil@temp
				\else
					\setbox\pgfutil@tempboxa=\hbox{{%  see \pgf@makehbox#1
						\pgfinterruptpicture% Interrupt picture to restore font
							\pgfutil@temp%
						\endpgfinterruptpicture%
					}}
					\pgfutil@tempdima=0.5\wd\pgfutil@tempboxa
					\advance\pgfutil@tempdima by \pgf@circ@res@right
					\advance\pgfutil@tempdima by 4pt
					\pgftext[x=\pgfutil@tempdima] {\pgfutil@temp}
				\fi
			\endpgfscope
		}
	}
}

\long\def\pgf@makehbox#1{%
  \setbox\pgf@hbox=\hbox{{%
    \pgfinterruptpicture%
      #1%
    \endpgfinterruptpicture%
    }}}



\pgfcircdeclaretube{diodetube}{}{} % shape diode already exists

\pgfcircdeclaretube{triode}
{
	\anchor{grid} {%
        \northwest
		\pgf@y=\ctikzvalof{tubes/grid shift}\pgf@y
		\
    }
}
{
	% Grid protrusion
	\pgf@xa=-\ctikzvalof{tubes/tube radius}\pgf@circ@res@right
	\advance\pgf@xa by -\ctikzvalof{tubes/grid protrusion}\pgf@circ@res@right
	\pgfpathmoveto{\pgfpoint{-\pgf@circ@res@right}{\ctikzvalof{tubes/grid shift}\pgf@circ@res@up}}
	\pgfpathlineto{\pgfpoint{\pgf@xa}{\ctikzvalof{tubes/grid shift}\pgf@circ@res@up}}
	% Grid dashes: calculations
	\pgf@xb=2\pgf@circ@res@right
	\pgf@circ@res@step=\ctikzvalof{tubes/tube radius}\pgf@xb
	\@tempcnta=\ctikzvalof{tubes/grid dashes}  % dashes*2+1
	\multiply\@tempcnta by 2\relax
	\advance\@tempcnta by 1\relax
	\advance\pgf@circ@res@step by -\pgf@xa
	\divide\pgf@circ@res@step by \@tempcnta
	% Grid dashes: draw
	\pgf@circ@res@temp=\pgf@xa
	\@tempcnta=\ctikzvalof{tubes/grid dashes}
	\loop
		\advance\pgf@circ@res@temp by\pgf@circ@res@step
		\pgfpathmoveto{\pgfpoint{\pgf@circ@res@temp}{\ctikzvalof{tubes/grid shift}\pgf@circ@res@up}}
		\advance\pgf@circ@res@temp by\pgf@circ@res@step
		\pgfpathlineto{\pgfpoint{\pgf@circ@res@temp}{\ctikzvalof{tubes/grid shift}\pgf@circ@res@up}}
		\advance\@tempcnta by-1
		\ifnum\@tempcnta>0\relax
	\repeat
}

\pgfcircdeclaretube{tetrode}
{
	\anchor{grid} {%
        \northwest
		\pgfutil@tempdima=\pgf@y
		\pgf@y=-\ctikzvalof{tubes/grid separation}\pgf@y
		\pgf@y=0.5\pgf@y
		\advance\pgf@y by \ctikzvalof{tubes/grid shift}\pgfutil@tempdima
    }
	\anchor{screen} {%
        \northwest
		\pgfutil@tempdima=\pgf@y
		\pgf@y=\ctikzvalof{tubes/grid separation}\pgf@y
		\pgf@y=0.5\pgf@y
		\advance\pgf@y by \ctikzvalof{tubes/grid shift}\pgfutil@tempdima
    }
}
{
	% Grid x/y points
	\pgf@xa=-\ctikzvalof{tubes/tube radius}\pgf@circ@res@right
	\advance\pgf@xa by -\ctikzvalof{tubes/grid protrusion}\pgf@circ@res@right
	\pgfutil@tempdima=\ctikzvalof{tubes/grid separation}\pgf@circ@res@up
	\pgfutil@tempdimb=-\pgfutil@tempdima
	\pgfutil@tempdima=0.5\pgfutil@tempdima
	\advance\pgfutil@tempdima by \ctikzvalof{tubes/grid shift}\pgf@circ@res@up
	\pgfutil@tempdimb=0.5\pgfutil@tempdimb
	\advance\pgfutil@tempdimb by \ctikzvalof{tubes/grid shift}\pgf@circ@res@up
	% Grid protrusion
	\pgfpathmoveto{\pgfpoint{-\pgf@circ@res@right}{\pgfutil@tempdimb}}
	\pgfpathlineto{\pgfpoint{\pgf@xa}{\pgfutil@tempdimb}}
	\pgfpathmoveto{\pgfpoint{-\pgf@circ@res@right}{\pgfutil@tempdima}}
	\pgfpathlineto{\pgfpoint{\pgf@xa}{\pgfutil@tempdima}}
	% Grid dashes: calculations
	\pgf@xb=2\pgf@circ@res@right
	\pgf@circ@res@step=\ctikzvalof{tubes/tube radius}\pgf@xb
	\@tempcnta=\ctikzvalof{tubes/grid dashes}  % dashes*2+1
	\multiply\@tempcnta by 2\relax
	\advance\@tempcnta by 1\relax
	\advance\pgf@circ@res@step by -\pgf@xa
	\divide\pgf@circ@res@step by \@tempcnta
	% Grid dashes: draw
	\pgf@circ@res@temp=\pgf@xa
	\@tempcnta=\ctikzvalof{tubes/grid dashes}
	\loop
		\advance\pgf@circ@res@temp by\pgf@circ@res@step
		\pgfpathmoveto{\pgfpoint{\pgf@circ@res@temp}{\pgfutil@tempdima}}
		\pgfpathlineto{\pgfpoint{\pgf@circ@res@temp+\pgf@circ@res@step}{\pgfutil@tempdima}}
		\pgfpathmoveto{\pgfpoint{\pgf@circ@res@temp}{\pgfutil@tempdimb}}
		\pgfpathlineto{\pgfpoint{\pgf@circ@res@temp+\pgf@circ@res@step}{\pgfutil@tempdimb}}
		\advance\pgf@circ@res@temp by\pgf@circ@res@step
		\advance\@tempcnta by-1
		\ifnum\@tempcnta>0\relax
	\repeat
}

\pgfcircdeclaretube{pentode}
{
	\anchor{grid} {%
        \northwest
		\pgfutil@tempdima=\pgf@y
		\pgf@y=-\ctikzvalof{tubes/grid separation}\pgf@y
		\advance\pgf@y by \ctikzvalof{tubes/grid shift}\pgfutil@tempdima
    }
	\anchor{screen} {%
        \northwest
		\pgf@y=\ctikzvalof{tubes/grid shift}\pgf@y
    }
	\anchor{suppressor} {%
        \northwest
		\pgfutil@tempdima=\pgf@y
		\pgf@y=\ctikzvalof{tubes/grid separation}\pgf@y
		\advance\pgf@y by \ctikzvalof{tubes/grid shift}\pgfutil@tempdima
    }
}
{
	% Grid x/y points
	\pgf@xa=-\ctikzvalof{tubes/tube radius}\pgf@circ@res@right
	\advance\pgf@xa by -\ctikzvalof{tubes/grid protrusion}\pgf@circ@res@right
	\pgfutil@tempdima=\ctikzvalof{tubes/grid separation}\pgf@circ@res@up
	\pgfutil@tempdimb=-\pgfutil@tempdima
	\advance\pgfutil@tempdima by \ctikzvalof{tubes/grid shift}\pgf@circ@res@up
	\advance\pgfutil@tempdimb by \ctikzvalof{tubes/grid shift}\pgf@circ@res@up
	\@tempdimc=\ctikzvalof{tubes/grid shift}\pgf@circ@res@up
	% Grid protrusion
	\pgfpathmoveto{\pgfpoint{-\pgf@circ@res@right}{\pgfutil@tempdimb}}
	\pgfpathlineto{\pgfpoint{\pgf@xa}{\pgfutil@tempdimb}}
	\pgfpathmoveto{\pgfpoint{-\pgf@circ@res@right}{\pgfutil@tempdima}}
	\pgfpathlineto{\pgfpoint{\pgf@xa}{\pgfutil@tempdima}}
	\pgfpathmoveto{\pgfpoint{-\pgf@circ@res@right}{\@tempdimc}}
	\pgfpathlineto{\pgfpoint{\pgf@xa}{\@tempdimc}}
	% Grid dashes: calculations
	\pgf@xb=2\pgf@circ@res@right
	\pgf@circ@res@step=\ctikzvalof{tubes/tube radius}\pgf@xb
	\@tempcnta=\ctikzvalof{tubes/grid dashes}  % dashes*2+1
	\multiply\@tempcnta by 2\relax
	\advance\@tempcnta by 1\relax
	\advance\pgf@circ@res@step by -\pgf@xa
	\divide\pgf@circ@res@step by \@tempcnta
	% Grid dashes: draw
	\pgf@circ@res@temp=\pgf@xa
	\@tempcnta=\ctikzvalof{tubes/grid dashes}
	\loop
		\advance\pgf@circ@res@temp by\pgf@circ@res@step
		\pgfpathmoveto{\pgfpoint{\pgf@circ@res@temp}{\pgfutil@tempdima}}
		\pgfpathlineto{\pgfpoint{\pgf@circ@res@temp+\pgf@circ@res@step}{\pgfutil@tempdima}}
		\pgfpathmoveto{\pgfpoint{\pgf@circ@res@temp}{\pgfutil@tempdimb}}
		\pgfpathlineto{\pgfpoint{\pgf@circ@res@temp+\pgf@circ@res@step}{\pgfutil@tempdimb}}
		\pgfpathmoveto{\pgfpoint{\pgf@circ@res@temp}{\@tempdimc}}
		\pgfpathlineto{\pgfpoint{\pgf@circ@res@temp+\pgf@circ@res@step}{\@tempdimc}}
		\advance\pgf@circ@res@temp by\pgf@circ@res@step
		\advance\@tempcnta by-1
		\ifnum\@tempcnta>0\relax
	\repeat
}


\makeatother
