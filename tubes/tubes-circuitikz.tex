%%
%% tubes-circuitikz.tex - test file for tubes in CircuiTiKZ
%%
%% Copyright (c)2019, Jesse E. J. op den Brouw
%%

%% This work may be distributed and/or modified under the
%% conditions of the LaTeX Project Public License, either version 1.3c
%% of this license or (at your option) any later version.
%% The latest version of this license is in
%%   http://www.latex-project.org/lppl.txt
%% and version 1.3c or later is part of all distributions of LaTeX 
%% version 2003/12/01 or later.

%% This work consists of the files tubes-circuitikz.tex,
%% tubes.tex, tubes-speedtest.tex

%% This software is provided 'as is', without warranty of any kind,
%% either expressed or implied, including, but not limited to, the
%% implied warranties of merchantability and fitness for a
%% particular purpose.

%% Jesse op den Brouw
%% Department of Electrical Engineering
%% The Hague University of Applied Sciences
%% Rotterdamseweg 137, 2628 AL, Delft
%% Netherlands
%% J.E.J.opdenBrouw@hhs.nl

%% The newest version of this document should always be available
%% from https://github.com/jesseopdenbrouw/circuitikz_enhancement

%% Version 1.0
%%

%%%%%%%%%%%%%%%%%%%%%%%%%%%%%%%%%%%%%%%%%%%%%%%%%%%%%%%%%%%%%%%%%%%%%%%%%%%%%%%
%%
%% CIRCUITIKZ - TUBES: DIODE, TRIODE, TETRODE and PENTODE
%%
%% (c)2019, J.E.J. op den Brouw <J.E.J.opdenBrouw@hhs.nl>
%%
%%%%%%%%%%%%%%%%%%%%%%%%%%%%%%%%%%%%%%%%%%%%%%%%%%%%%%%%%%%%%%%%%%%%%%%%%%%%%%%

\makeatletter

\pgfcircdeclaretube{pentode suppressor to cathode}
{
	\anchor{grid} {%
        \northwest
		\pgfutil@tempdima=\pgf@y
		\pgf@y=-\ctikzvalof{tubes/grid separation}\pgf@y
		\advance\pgf@y by \ctikzvalof{tubes/grid shift}\pgfutil@tempdima
    }
	\anchor{screen} {%
        \northwest
		\pgf@y=\ctikzvalof{tubes/grid shift}\pgf@y
    }
}
{
	% Grid x/y points
	\pgf@xa=-\ctikzvalof{tubes/tube radius}\pgf@circ@res@right
	\advance\pgf@xa by -\ctikzvalof{tubes/grid protrusion}\pgf@circ@res@right
	\pgfutil@tempdima=\ctikzvalof{tubes/grid separation}\pgf@circ@res@up
	\pgfutil@tempdimb=-\pgfutil@tempdima
	\advance\pgfutil@tempdima by \ctikzvalof{tubes/grid shift}\pgf@circ@res@up
	\advance\pgfutil@tempdimb by \ctikzvalof{tubes/grid shift}\pgf@circ@res@up
	\@tempdimc=\ctikzvalof{tubes/grid shift}\pgf@circ@res@up
	% Grid protrusion
	\pgfpathmoveto{\pgfpoint{-\pgf@circ@res@right}{\pgfutil@tempdimb}}
	\pgfpathlineto{\pgfpoint{\pgf@xa}{\pgfutil@tempdimb}}
	\pgfpathmoveto{\pgfpoint{-\pgf@circ@res@right}{\@tempdimc}}
	\pgfpathlineto{\pgfpoint{\pgf@xa}{\@tempdimc}}
	% Grid dashes: calculations
	\pgf@xb=2\pgf@circ@res@right
	\pgf@circ@res@step=\ctikzvalof{tubes/tube radius}\pgf@xb
	\@tempcnta=\ctikzvalof{tubes/grid dashes}  % dashes*2+1
	\multiply\@tempcnta by 2\relax
	\advance\@tempcnta by 1\relax
	\advance\pgf@circ@res@step by -\pgf@xa
	\divide\pgf@circ@res@step by \@tempcnta
	% Grid dashes: draw
	\pgf@circ@res@temp=\pgf@xa
	\@tempcnta=\ctikzvalof{tubes/grid dashes}
	\loop
		\advance\pgf@circ@res@temp by\pgf@circ@res@step
		\ifnum\@tempcnta>1\relax
			\pgfpathmoveto{\pgfpoint{\pgf@circ@res@temp}{\pgfutil@tempdimb}}
			\pgfpathlineto{\pgfpoint{\pgf@circ@res@temp+\pgf@circ@res@step}{\pgfutil@tempdimb}}
			\pgfpathmoveto{\pgfpoint{\pgf@circ@res@temp}{\@tempdimc}}
			\pgfpathlineto{\pgfpoint{\pgf@circ@res@temp+\pgf@circ@res@step}{\@tempdimc}}
		\fi
		\pgfpathmoveto{\pgfpoint{\pgf@circ@res@temp}{\pgfutil@tempdima}}
		\pgfpathlineto{\pgfpoint{\pgf@circ@res@temp+\pgf@circ@res@step}{\pgfutil@tempdima}}
		\advance\pgf@circ@res@temp by\pgf@circ@res@step
		\advance\@tempcnta by-1
		\ifnum\@tempcnta>0\relax
	\repeat
	% Grid: connection from suppressor to cathode
	\pgfsetcornersarced{\pgfpoint{\ctikzvalof{tubes/cathode corners}\pgf@circ@res@up}{\ctikzvalof{tubes/cathode corners}\pgf@circ@res@up}}
	\pgfpathlineto{\pgfpoint{\pgf@circ@res@temp}{\pgfutil@tempdima-2*\ctikzvalof{tubes/grid separation}\pgf@circ@res@up}}
	\pgfpathlineto{\pgfpoint{\ctikzvalof{tubes/cathode width}\pgf@circ@res@right-0.4142136*\ctikzvalof{tubes/cathode corners}\pgf@circ@res@right}{-\ctikzvalof{tubes/cathode distance}\pgf@circ@res@up-0.4142136*\ctikzvalof{tubes/cathode corners}\pgf@circ@res@up}}
					
}

\pgfdeclareshape{mycircle}
{
    \savedanchor\centerpoint{
        \pgf@x = .5\wd\pgfnodeparttextbox
        \pgf@y = .5\ht\pgfnodeparttextbox
    }
    \anchor{center}{\centerpoint}

    \backgroundpath{
        \pgfnode{triode}{center}{}{}{\pgfusepath{draw}}
    }

}

\makeatother
